\documentclass[]{article}

%opening
\title{Managing Big Data - Assignment 8}
\author{Joost van Doorn, Bas Janssen and Niek Tax}

\begin{document}

\maketitle

\section*{Assignment 8.1}
A search within the proceedings of SIGIR inside the ACM Digital Library resulted in 33 results for \texttt{'mapreduce'}, 14 results for \texttt{'hadoop'} and 180 results in \texttt{'big data'}. We did find some papers about sentiment analysis: most papers are related to creating or optimizing sentiment analysis or sentiment classification. However, we also found papers who have exactly the same idea as we have: relate sales performance and movie sentiment on the web. The most important papers (we define the most important papers as papers with much relevance with out idea and with the most citations) are \texttt{ARSA: a sentiment-aware model for predicting sales performance using blogs}\cite{Liu2007} and \texttt{"Knowledge transformation for cross-domain sentiment classification"}\cite{Li2009}. We also found the paper by \cite{Liu2007} during our initial literature review for assignment 6.

\section*{Assignment 8.2}
A search on \texttt{'mapreduce'} resulted in 2700 results, \texttt{'hadoop'} resulted in about 2000 results and \texttt{'big data'} in 30700+ results. Apart from the results mentioned in the previous assignment we found \texttt{S-PLSA + : Adaptive Sentiment Analysis with Application to Sales Performance Prediction}\cite{Liu} and \texttt{A quality-aware model for sales prediction using reviews}. \cite{Yu2010}. The importance of the paper was determined in the same way as it was in assignment 8.1.

\section*{Assignment 8.3}
The papers found (and described above) contained a lot of information about the creation and training of sentiment analysis models. Many of the papers try to model a system to predict movie sales before the final sales are known, however we only tried to check if we could find a correlation on movies sales and sentiment using a simple algorithm. Because we used the trained models of the Stanford CoreNLP, the improvement of the tool or training of the model is clearly out of scope. Of course, our algorithm is very primitive: we are not sure whether a sentence is referring to a movie and we don't limit our data source (we do not filter our data on time or source): there are lots of aspects which can be improved. Many solutions, algorithm optimizations and approaches mentioned in the papers, like the optimizations mentioned above, could be used in our own model.

\section*{Assignment 8.4-Assignment 8.5}
The report can be found in the attachment.



\bibliographystyle{apalike}
\bibliography{assignment8}

\end{document}
